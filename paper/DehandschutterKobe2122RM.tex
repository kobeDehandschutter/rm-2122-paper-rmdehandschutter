%==============================================================================
% Sjabloon onderzoeksvoorstel bachelorproef
%==============================================================================%
% Compileren in TeXstudio:
%
% - Zorg dat Biber de bibliografie compileert (en niet Biblatex)
%   Options > Configure > Build > Default Bibliography Tool: "txs:///biber"
% - F5 om te compileren en het resultaat te bekijken.
% - Als de bibliografie niet zichtbaar is, probeer dan F5 - F8 - F5
%   Met F8 compileer je de bibliografie apart.
%
% Als je JabRef gebruikt voor het bijhouden van de bibliografie, zorg dan
% dat je in ``biblatex''-modus opslaat: File > Switch to BibLaTeX mode.

\documentclass{hogent-article}

\usepackage{lipsum} % Voor vultekst

%------------------------------------------------------------------------------
% Metadata over het artikel
%------------------------------------------------------------------------------

%---------- Titel & auteur ----------------------------------------------------

% TODO: (fase 2) geef werktitel van je eigen voorstel op
\PaperTitle{Vergelijkende studie tussen verschillende spraakassistenten}
% Dit is typisch de opdracht en het vak waarvoor dit artikel geschreven is, bv.
% ``Verslag onderzoeksproject Onderzoekstechnieken 2018-2019''
\PaperType{Paper Research Methods: onderzoeksvoorstel}

% TODO: (fase 1) vul je eigen naam in als auteur, geef ook je emailadres mee!
\Authors{Kobe Dehandschutter\textsuperscript{1}} % Authors
% Als het hier effectief gaat om een voorstel voor de bachelorproef, dan ben je
% hier verplicht de naam van je co-promotor in te vullen. Zoniet, dan kan je het
% leeg laten.
\CoPromotor{}

% Contactinfo: Geef hier de contactgegevens van elke auteur van het artikel (en
% indien van toepassing ook van de co-promotor).
\affiliation{
  \textsuperscript{1} \href{mailto:kobe.dehandschutter@student.hogent.be}{kobe.dehandschutter@student.hogent.be}
}

%---------- Abstract ----------------------------------------------------------

\Abstract{% TODO: (fase 6)
Spraakassistenten worden wereldwijd meer en meer ingeburgerd. Ze leveren namelijk heel wat gebruikersgemak op. In dit onderzoek wordt een vergelijkende studie uitgevoerd tussen de vier meest gebruikte spraakassistenten: Siri, Alexa, Bixby en Google Assistant. De eerste factor die vergeleken wordt, is de privacy. Bij velen vermindert de bezorgdheid hierover, maar is dat ook terecht? Daarnaast is het belangrijk dat de assistent accuraat en actueel antwoordt. Dit wordt vergeleken in een tweede onderdeel. Vervolgens wordt bestudeerd hoe persoonlijk iedere assistent is. Als laatste wordt gekeken naar hoe goed de spraakassistenten om kunnen met mensen die stotteren. Het is nog steeds een probleem dat de assistent deze mensen vaak onderbreekt of stopt met luisteren. Er wordt verondersteld dat elke assistent zijn eigen voordelen zal hebben en er dus geen ultieme winnaar aangeduid zal kunnen worden. Er zal echter wel per vergelijking een score worden bijgehouden, waardoor er op elk vlak zal kunnen geconcludeerd worden welke assistent de bovenhand neemt.

}

%---------- Onderzoeksdomein en sleutelwoorden --------------------------------
% TODO: (fase 2) Vul de sleutelwoorden aan.

% Het eerste sleutelwoord beschrijft het onderzoeksdomein. Je kan kiezen uit
% deze lijst:
%
% - Mobiele applicatieontwikkeling
% - Webapplicatieontwikkeling
% - Applicatieontwikkeling (andere)
% - Systeembeheer
% - Netwerkbeheer
% - Mainframe
% - E-business'
% - Databanken en big data
% - Machineleertechnieken en kunstmatige intelligentie
% - Andere (specifieer)
%
% De andere sleutelwoorden zijn vrij te kiezen.

\Keywords{Applicatieontwikkeling (andere); voice AI; Siri; Alexa}
\newcommand{\keywordname}{Sleutelwoorden} % Defines the keywords heading name

%---------- Titel, inhoud -----------------------------------------------------

\begin{document}

\flushbottom % Makes all text pages the same height
\maketitle % Print the title and abstract box
\tableofcontents % Print the contents section
\thispagestyle{empty} % Removes page numbering from the first page

%------------------------------------------------------------------------------
% Hoofdtekst
%------------------------------------------------------------------------------

\section{Inleiding}

% TODO: (fase 2) introduceer je gekozen onderwerp, formuleer de onderzoeksvraag en deelvragen. Wat is de doelstelling (is die S.M.A.R.T.?), wat zal het resultaat zijn van het onderzoek (een Proof-of-Concept, een prototype, een advies, ...)? Waarom is het nuttig om dit onderwerp te onderzoeken?


Een spraakassistent is een functionaliteit die de laatste jaren aan een enorme opmars bezig is.
Er bestaan er ondertussen dan ook al veel verschillende. De meest bekende uit die lijst zijn Siri en Alexa.
Volgens een studie, uitgevoerd door mevrouw Federica Laricchia, waren er wereldwijd in 2020 al zo'n 4,2 miljard spraakassistenten in gebruik. Laricchia voorspelt dat dit aantal zal verdubbeld zijn tegen 2024, zodat er op dat moment meer spraakassistenten zullen bestaan dan mensen \autocite{Laricchia2022}.
\\\indent
In dit onderzoek worden de vier grootste spraakassistenten vergeleken, deze zijn: Siri (van Apple), Alexa (van Amazon), Google Assistant (van Google) en Bixby (van Samsung). Ze worden onder de loep genomen op het vlak van privacy, persoonlijkheid, hoe accuraat en actueel hun antwoord is en hoe goed ze overweg kunnen met mensen met een spraakgebrek. Aangezien Microsoft ook één van de grootste technologie bedrijven is, werd eraan gedacht om ook hun spraakassistent (Cortana) op te nemen in het onderzoek. Echter is de ontwikkeling daarvan in maart 2021 stopgezet omdat het niet de gewenste inpakt had volgens Microsoft. Het bedrijf realiseerde zich ook dat Cortana de andere grote spraakassistenten niet kon evenaren \autocite{Tiwari2021}.
\\\indent
Volgens \textcite{Benke2021} is spraaktechnologie ongetwijfeld de volgende grote stap vooruit in de informatica wereld. Deze technologie heeft dan ook veel voordelen. Zo verbetert de klantbetrokkenheid enorm aangezien spraakassistenten steeds beter worden in het leren kennen van een persoon. Hierdoor krijgen de klanten meer persoonlijke hulp. Daarnaast is het ook een stuk gemakkelijker om de juiste informatie te verkrijgen door verbale communicatie, dan door alles te moeten typen. Bovendien heeft een spraakassistent geen nine-to-five job, maar kan deze 24/7/365 klantenservice leveren \autocite{Benke2021}.

\footnote{https://github.com/kobeDehandschutter/rm-2122-paper-rmdehandschutter}
\section{Overzicht literatuur}

% TODO: (fase 4) schrijf de literatuurstudie uit en gebruik waar gepast referenties naar de vakliteratuur.

% Refereren naar de literatuur kan met:
% \autocite{BIBTEXKEY} -> (Auteur, jaartal)
% \textcite{BIBTEXKEY} -> Auteur (jaartal)


Het is geen verrassing dat het gebruik van spraakassistenten de laatste jaren een forse stijging heeft ondergaan. Het zijn nog steeds vooral de jongeren (mensen onder de 34 jaar) die er gebruik van maken, maar verrassend genoeg blijkt uit een onderzoek van DirectResearch dat in Nederland de grootste stijging in gebruik te zien was bij de 65-plussers \autocite{Molkenboer2022}.
Volgens Jeannetta Berghahn van DirectResearch is de grootste reden voor deze stijging het feit dat door middel van spraaktechnologie, de complexe technologie wegvalt. Dit is een groot voordeel voor ouderen die al iets vaker moeite hebben met technologie \autocite{Molkenboer2022}.
\\\indent
Uit hetzelfde onderzoek blijkt dat Nederlanders zich steeds minder zorgen maken over hun privacy wanneer ze gebruik maken van een spraakassistent. In 2019 had zo'n 27\% het hier moeilijk mee. Dit is in 2021 gezakt naar 19\% \autocite{Molkenboer2022}. Aangezien er steeds meer vertrouwen komt in de spraaktechnologie stellen de Nederlanders ook hogere eisen aan hun spraakassistent. Een derde van de respondenten van het onderzoek geeft aan dat ze vaker een spraakassistent zouden gebruiken, mocht de kwaliteit verbeteren. Het is voor hen belangrijk dat ze hun vraag of commando niet meerdere keren moeten herhalen \autocite{Molkenboer2022}.
\\\indent
Er zijn jammer genoeg ook mensen die niet kunnen genieten van de voordelen van spraakassistenten. Namelijk mensen met een spraakgebrek, meer bepaald mensen die stotteren. Volgens \textcite{Fitzpatrick2020}, die zelf stottert sinds haar geboorte, is Alexa juist wat ze niet nodig heeft. Iedere keer dat ze opnieuw probeert Alexa te gebruiken, wordt ze eraan herinnerd dat ze haar niet begrijpt. Spraakassistenten zijn niet voorbereid op het herhaaldelijk dezelfde letter horen, waardoor ze gewoon stoppen met luisteren \autocite{Fitzpatrick2020}. Een eerste stap in de goede richting is de hold-to-talk functie, zodat men zelf kan beslissen wanneer het commando gedaan is en de spraakassistent niet onderbreekt. Apple heeft echter al laten weten dat ze bezig zijn met Siri te trainen aan de hand van een databank met 28.000 spraakberichten van stotteraars \autocite{Dormehl2021}. Hopelijk wordt dit een grote stap in de goede richting.
\\\indent
De coronacrisis heeft ook geholpen om enkele grote voordelen van spraaktechnologie te ontdekken. Onderzoekers van de Carnegie Mellon universiteit in Pittsburgh maakten in 2020 een applicatie die met gebruik van spraaktechnologie de stem en ademtechnieken analyseerde van duizenden mensen die positief getest hadden op Covid-19. De app was uiteindelijk in staat om te detecteren of een persoon wel of niet besmet was met Covid-19 \autocite{Gujral2021}.
\\\indent
Het blijft niet enkel bij Corona: spraaktechnologie speelt momenteel een belangrijke rol in het helpen van mensen met mentale gezondheidsproblemen. Er wordt geschat dat er wereldwijd zo'n 40 miljoen mensen per jaar lijden aan een mentale ziekte. Spraaktechnologie wordt momenteel gebruikt om stressfactoren te identificeren en te helpen om die stress te verminderen. Het wordt ook gebruikt om zelfmoord risico's in te schatten en hulp te bieden in ziekenhuizen en mentale gezondheidsklinieken \autocite{Gujral2021}.

\subsection{Meer informatie over de spraakassistenten}
Hier wordt kort wat achtergrondinformatie gegeven over alle verschillende spraakassistenten die worden vergeleken in deze studie.
\subsubsection{Siri}
Siri werd gecreëerd door Adam Cheyer. In 1993 maakte hij een eerste versie en in de twintig jaar die daarop volgden, maakte hij nog 50 nieuwe versies. Cheyer richtte dan samen met Tom Gruber en Dag Kittlaus het bedrijf Siri Inc op. In 2010 werd Siri geïntroduceerd als een app, maar Apple wou liever het volledige bedrijf inclusief de technologie overkopen. Steve Jobs nam toen persoonlijk contact op met het bedrijf, maar na de eerste ontmoeting wilden ze het nog niet verkopen. Enkele maanden later werd het bedrijf dan toch overgekocht door Apple voor iets meer dan 200 miljoen dollar. Siri is momenteel één van de hoogste prioriteiten bij Apple \autocite{Randy}.


\subsubsection{Alexa}
Amazon heeft een slimme speaker (Amazon Echo) waarvan het originele idee hierachter van Jeff Bezos (oprichter en CEO van Amazon) zelf kwam. In 2011 schetste hij een apparaat dat volledig bestuurbaar was met de stem. Bezos stelde een team samen om dit idee tot werkelijkheid te brengen, met Greg Hart als kopman. Dit team had echter geen kennis van spraaktechnologie, dus ging het op zoek naar een stem voor in het apparaat. Deze werd gevonden in Polen. Het bedrijf Ivona, opgericht in 2001 door Lukasz Osowski, begon met het verzamelen van spraakberichten, ingesproken door Jacek Labijak, een Poolse acteur. Aan de hand van deze berichten maakten ze een databank vol geluiden die Labijak maakte om zo zinnen te vormen die de acteur nooit gezegd had. Vanaf 2006 nam Ivona deel aan een competitie voor de meest natuurlijke computerstem en ze wonnen ook enkele keren. Hierdoor werden ze ontdekt door het team van Hart. In 2012 werd Ivona overgekocht voor 30 miljoen dollar. De Polen mochten er wel verder aan blijven werken onder Amazon. De aankoop werd nog een jaar geheimgehouden totdat Alexa op punt stond \autocite{Stone2021}.

\subsubsection{Google Assistant}
Het verhaal achter Google Assistant begon in 2011. De eerste versie kreeg de naam Google Voice Search en hoewel het voor die tijd enorm vernieuwend was om een gsm commando's te geven met gebruik van de stem, waren de functionaliteiten ervan nog zeer eenvoudig. Zo kon de assistent nog geen antwoord geven. De volgende grote versie was Google Now. Deze kon al heel wat meer, met als grootste aanpassingen: het feit dat de assistent ook terug sprak en het begin van 'OK Google' (het commando om de assistent te laten luisteren). In 2017 kwam dan voor het eerst de huidige versie van Google Assistant. Deze werd geïntegreerd in (en vanaf) Android 6.0 en in de slimme speaker: Google Home \autocite{Jansen2018}.

\subsubsection{Bixby}
Bixby is ontworpen en ontwikkeld door Samsung Electronics en had eerst de naam 'S Voice'. In 2017 heeft Samsung 50 miljoen dollar betaald aan het bedrijf Innoetics, gespecialiseerd in text-to-speech en voice-to-speech technologie, om S Voice om te zetten naar Bixby, één van de beste spraakassistenten \autocite{Reigh2017}. In 2018 hebben ze een versie uitgebracht die naast hun smartphones ook te gebruiken was in hun andere apparaten zoals televisietoestellen en frigo's \autocite{Wikipedia2022}.

\section{Methodologie}

% TODO: (fase 5) beschrijf in detail in welke fasen je onderzoek uiteenvalt, hoe lang elke fase duurt en wat het concrete resultaat van elke fase is. Welke onderzoekstechniek ga je toepassen om elk van je onderzoeksvragen te beantwoorden? Gebruik je hiervoor experimenten, vragenlijsten, simulaties? Je beschrijft ook al welke tools je denkt hiervoor te gebruiken of te ontwikkelen.

Het onderzoek bestaat uit 5 verschillende fases. In elke fase wordt een score bijgehouden per assistent.

Eén van de belangrijkste eigenschappen van een spraakassistent is de privacy ervan. Zo beweert Apple dat Siri de sterkste privacy heeft van alle intelligente assistenten \autocite{Apple}. In de eerste fase worden de spraakassistenten dus vergeleken op vlak van hun privacy. Het is zeer belangrijk dat persoonlijke informatie die gedeeld wordt met de assistent niet verkeerd gebruikt kan worden door het bedrijf achter de assistent. Daarnaast is het nog belangrijker dat die informatie niet kan onderschept worden door anderen met eventuele slechte bedoelingen. Deze vergelijkingen worden gedaan door middel van de privacyverklaringen te analyseren om zo te ontdekken welke informatie er effectief bijgehouden of zelfs gebruikt wordt. Hierna kan worden geconcludeerd welke assistent de sterkste privacy heeft.
\\\indent
De tweede fase vergelijkt hoe goed de assistenten commando's uitvoeren en hoe correct en actueel hun antwoord is. Dit wordt gedaan door een lijst aan commando's te maken. Zowel makkelijke zaken zoals 'Zet een alarm om 7 uur 's ochtends' als moeilijkere zaken zoals 'Verhoog de helderheid van het scherm tot 95\%'. Naast deze commando's zal er ook een lijst actuele vragen gesteld worden. Een simpel voorbeeld hiervan is de vraag: 'Wint Rusland van Oekraïne?'. Wie de actualiteit volgt zal weten dat de vraag gaat over de oorlog tussen de twee landen, maar toen deze vraag aan Siri werd gesteld, antwoordde ze met de resultaten van een voetbalmatch. In maart 2020 vroeg \textcite{Heilweil2020} aan Alexa hoe ze getest kon worden voor het coronavirus. Alexa antwoordde correct, maar toen ze dezelfde vraag aan Siri stelde, adviseerde deze een paar links en één daarvan was voor het MERS virus. Dit is ook een coronavirus, maar niet het actuele virus natuurlijk \autocite{Heilweil2020}. Na alle vragen op deze twee lijsten aan elke spraakassistent te hebben gesteld, kan geconcludeerd worden welke assistent het best commando's kan uitvoeren en het meest correct kan antwoorden.
\\\indent
In de derde fase wordt onderzocht hoe persoonlijk de assistenten zijn. Er wordt een lijst van persoonlijke gegevens voorgesteld aan elke assistent. In die gegevens zitten relaties (zoals broer, moeder en nicht), maar ook adressen, lievelingsgerechten, hobby's... Vervolgens worden commando's gegeven zoals: 'Bel mijn broer' of 'Stel de gps in naar huis' en zal blijken hoe goed elke assistent omgaat met persoonlijke informatie.
\\\indent
Hoe goed een assistent overweg kan met iemand die stottert wordt in de vierde fase vergeleken. Er wordt nagegaan welke assistenten een hold-to-talk functie voorzien zodat de persoon niet constant onderbroken wordt. Dit is echter niet genoeg. De assistent moet ook overweg kunnen met het feit dat bijvoorbeeld een 'T' klank ongewild vijf keer na elkaar voorkomt. Ook in deze fase wordt er een lijst van commando's opgesteld. Deze worden voorgelezen door iemand die stottert. Hierbij wordt gekeken of de assistent correct antwoordt en/of het commando uitvoert.
\\\indent
In de vijfde en laatste fase worden alle resultaten van de vorige fases geanalyseerd en beschreven. In iedere fase werd een score per assistent bijgehouden. Deze score verhoogde per correct uitgevoerd commando. Hieruit kan vervolgens geconcludeerd worden welke spraakassistent op welk vlak de betere is.

\section{Verwachte conclusies}

% TODO: (fase 6) beschrijf wat je verwacht uit je onderzoek en waarom (bv. volgens je literatuuronderzoek is softwarepakket A het meest gebruikte en denk je dat het voor deze casus ook het meest geschikt zal zijn). Natuurlijk kan je niet in de toekomst kijken en mag je geen alternatieve mogelijkheden uitsluiten. In de praktijk gebeurt het ook vaak dat een onderzoek tot verrassende resultaten leidt, dat maakt het proces nog interessanter!


Er wordt verwacht dat iedere assistent zijn eigen kwaliteiten heeft en er dus geen van de vier er met kop en schouders bovenuit steekt. Qua privacy beweert Apple dat Siri het sterkst is en dit wordt daarom ook verwacht. Op vlak van correct antwoorden en commando's uitvoeren wordt vermoed dat Google Assistant de bovenhand neemt. Als er dan wordt gekeken naar actuele vragen, dan wordt verwacht dat ze allemaal heel goed gaan scoren aangezien ze veel opzoeken op het internet. Het verschil zal hier waarschijnlijk heel klein zijn, maar er wordt toch verwacht dat Google Assistant en Alexa het iets beter zullen doen. Als er wordt vergeleken op persoonlijkheid, wordt verwacht dat Siri de hoogste score krijgt. In de laatste vergelijking (begrijpen van stotteraars) wordt verwacht dat Siri en Bixby een hold-to-talk functie hebben, terwijl Alexa en Google Assistant meer inzetten op het verstaan van stotteraars.

%------------------------------------------------------------------------------
% Referentielijst
%------------------------------------------------------------------------------
% TODO: (fase 4) de gerefereerde werken moeten in BibTeX-bestand
% bibliografie.bib voorkomen. Gebruik JabRef om je bibliografie bij te
% houden.

\phantomsection
\printbibliography[heading=bibintoc]

\end{document}
